\subsection{Upravaljanje zalihama lekova i jednokratne opreme} 
\indent Ovaj slučaj upotrebe omogućava praćenje trenutnog stanja zaliha lekova i medicinskog potrošnog materijala u veterinarskoj klinici. Sistem prati potrošnju, upozorava na moguće nestašice i omogućava blagovremeno kreiranje narudžbina kako bi se održala neprekidna dostupnost potrebnih sredstava za rad. \\

\begin{figure}[h]
    \centering
    \includegraphics[width=0.7\linewidth]{dijagram3.png}
    \caption{Slučaj upotrebe: Upravljanje zalihama lekova i jednokratn}
    \label{fig:placeholder}
\end{figure}

\subsubsection{Evidencija potrosnje}
\indent \textbf{Kratak opis} - Veterinar belezi kada iskoristi odredjeni lijek ili materijal.\\
\indent \textbf{Akteri} - Veterinar\\
\indent \textbf{Preduslovi} - Veterinar je ulogovan u sistem i ima odgovarajuće privilegije za unos potrošnje. U sistemu postoji katalog lekova i medicinskog materijala.
\indent \textbf{Postuslov} - Stanje zaliha za izabrani artikal je ažurirano. Ako je artikal ispod minimuma, sistem evidentira obaveštenje za menadžera. \\
\indent \textbf{Glavni tok:}
\begin{enumerate}
    \item Veterinar otvara katalog lijekova i opreme. 
    \item Veterinar bira artikal koji je koristio i unosi kolicinu koja je potrosena.
    \item Veterinar potvrdjuje unos.
    \item Sistem proverava da li je unos validan i ažurira stanje zaliha.
    \item Ako novo stanje padne ispod definisanog minimuma, sistem generiše obaveštenje za menadžera.
    \item Veterinar ponavlja korake 2-5 za svaki dodatni artikal. 
\end{enumerate}
\indent \textbf{Alternativni tok:}
\begin{enumerate}
    \item[A1.]\textbf{Uneta kolicina je veca od dostupne zalihe} Sistem odbija transakciju i prikazuje odgovarajuce obavestenje.
    \item[A2.]\textbf{Veterinar predlaze novi artikal za narudzbinu} Veterinar unosi potrošnju i primeti da je potreban artikal koji se obično ne naručuje ili nije u katalogu. Veterinar šalje predlog menadžeru putem sistema (notifikacija ili lista predloga).
\end{enumerate}

\subsubsection{Kreiranje narudzbine}
\indent \textbf{Kratak opis} - Menadžer prima obaveštenja o artiklima koje treba naručiti - bilo da je zaliha ispod minimuma ili je veterinar predložio novi artikal - i kreira narudžbenicu na osnovu tih informacija.\\
\indent \textbf{Akteri} - Menadzer\\
\indent \textbf{Preduslovi} - Menadžer je ulogovan u sistem. Sistem vodi evidenciju zaliha i potrošnje. Sistem može primati predloge od veterinara i detektovati artikle ispod minimuma.\\
\indent \textbf{Postuslov} - Narudžbenica je kreirana i evidentirana u sistemu. \\
\indent \textbf{Glavni tok:}
\begin{enumerate}
    \item Sistem generiše obaveštenja menadžeru o artiklima čija je zaliha ispod minimuma i/o predlozima od veterinara za artikle koji ranije nisu naručivani.
    \item Menadžer otvara obaveštenja i pristupa listi artikala.
    \item Menadžer bira koje artikle želi da naruči.
    \item Sistem, po potrebi, predlaže količinu na osnovu istorije potrošnje.
    \item Sistem proverava budžet i prikazuje upozorenje ako je prekoračen.
    \item Menadžer potvrđuje ili menja količine.
    \item Menadžer kreira narudžbenicu, šalje je dobavljaču i sistem je evidentira.
\end{enumerate}
\indent \textbf{Alternativni tok:}
\begin{enumerate}
    \item[A1.]\textbf{Menadžer želi da doda artikle koji nisu u obaveštenjima} Menadžer otvara katalog i ručno bira dodatne artikle za narudžbenicu.
    \item[A2.]\textbf{Artikal koji želi da naruči ne postoji u katalogu} Menadžer može ručno dodati artikal u narudžbenicu i uneti sve potrebne podatke (novi artikal se formalno uvodi u katalog kada stigne).
\end{enumerate}

\subsubsection{Prijem robe}
\indent \textbf{Kratak opis} - Menadžer prima naručenu robu u kliniku i ažurira stanje zaliha u katalogu, uključujući beleženje količina i datuma prijema. \\
\indent \textbf{Akteri} - Menadzer\\
\indent \textbf{Preduslovi} - Narudžbenica je kreirana i roba je stigla u kliniku. Menadžer je ulogovan u sistem.\\
\indent \textbf{Postuslov} - Katalog dostupnih lekova i materijala je ažuriran. Sistem evidentira primljene količine i datume prijema. \\
\indent \textbf{Glavni tok:}
\begin{enumerate}
    \item Menadzer otvara katalog na svom nalogu.
    \item Sistem prikazuje listu narudžbenica spremnih za prijem.
    \item Menadžer bira narudžbenicu koju prima.
    \item Menadžer unosi stvarno primljene količine za svaki artikal.
    \item Sistem proverava da li unete količine odgovaraju naručenim i upozorava u slučaju odstupanja.
    \item Sistem ažurira stanje zaliha u katalogu i automatski evidentira datum prijema i primljene količine.
\end{enumerate}
\indent \textbf{Alternativni tok:}
\begin{enumerate}
    \item[A1.]\textbf{Stiglo je manje robe nego naruceno} Sistem prikazuje upozorenje. Menadžer može ponovo naručiti razliku.
    \item[A2.] \textbf{Stiglo je vise robe nego naruceno} Sistem prikazuje upozorenje. Menadžer može ili dodati višak u katalog kao dostupnu zalihu ili kontaktirati dobavljača radi korekcije.
\end{enumerate}
