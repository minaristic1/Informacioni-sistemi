\subsection{Upravaljanje zalihama lekova i jednokratne opreme} 
\indent Ovaj slučaj upotrebe omogućava praćenje trenutnog stanja zaliha lekova i medicinskog potrošnog materijala u veterinarskoj klinici. Sistem prati potrošnju, upozorava na moguće nestašice i omogućava blagovremeno kreiranje narudžbina kako bi se održala neprekidna dostupnost potrebnih sredstava za rad. \\

\begin{figure}[h]
    \centering
    \includegraphics[width=0.7\linewidth]{dijagram3.png}
    \caption{Slučaj upotrebe: Upravljanje zalihama lekova i jednokratn}
    \label{fig:placeholder}
\end{figure}

\subsubsection{Evidencija potrosnje}
\indent \textbf{Kratak opis} - Veterinar belezi kada iskoristi odredjeni lijek ili materijal.\\
\indent \textbf{Akteri} - Veterinar\\
\indent \textbf{Preduslovi} - Veterinar je ulogovan u sistem i ima odgovarajuće privilegije za unos potrošnje. U sistemu postoji katalog lekova i medicinskog materijala.
\indent \textbf{Postuslov} - Stanje zaliha za izabrani artikal je ažurirano. Ako je artikal ispod minimuma, sistem evidentira obaveštenje za menadžera. \\
\indent \textbf{Glavni tok:}
\begin{enumerate}
    \item Veterinar otvara katalog lijekova i opreme. 
    \item Veterinar bira artikal koji je koristio i unosi kolicinu koja je potrosena. 
    \item Veterinar potvrdjuje unos. 
    \item Sistem proverava da li je unos validan i ažurira stanje zaliha. Ako količina premašuje raspoloživu zalihu, ide se u alternativni tok A1. 
    Ako traženi artikal nije u katalogu, ide se u A2.
    \item Ako novo stanje padne ispod definisanog minimuma, sistem generiše obaveštenje za menadžera.
    \item Veterinar ponavlja korake 2-5 ukoliko ima jos artikala koje je koristio. Ako nema, veterinar zavrsava unos. 
\end{enumerate}
\indent \textbf{Alternativni tok:}
\begin{enumerate}
    \item[A1.]\textbf{Uneta kolicina je veca od dostupne zalihe} 
    \begin{enumerate}
        \item[1.] Sistem odbija unos i prikazuje poruku o nedovoljnoj zalihi.
        \item[2.] Sistem vraća veterinara na izbor artikla (glavni tok, korak 2).
    \end{enumerate}
    \item[A2.]\textbf{Veterinar predlaze novi artikal za narudzbinu} 
        \begin{enumerate}
            \item[1.] Veterinar bira opciju „Predlog nabavke“.
            \item[2.] Sistem šalje obaveštenje menadžeru ili dodaje stavku u listu predloga.
            \item[3.] Veterinar se vraća na glavni tok i nastavlja korak 2 za neki drugi artikal.
        \end{enumerate}
\end{enumerate}

\subsubsection{Kreiranje narudzbine}
\indent \textbf{Kratak opis} - Menadžer prima obaveštenja o artiklima koje treba naručiti - bilo da je zaliha ispod minimuma ili je veterinar predložio novi artikal - i kreira narudžbenicu na osnovu tih informacija.\\
\indent \textbf{Akteri} - Menadzer\\
\indent \textbf{Preduslovi} - Menadžer je ulogovan u sistem. Sistem vodi evidenciju zaliha i potrošnje. Sistem može primati predloge od veterinara i detektovati artikle ispod minimuma.\\
\indent \textbf{Postuslov} - Narudžbenica je kreirana i evidentirana u sistemu. \\
\indent \textbf{Glavni tok:}
\begin{enumerate}
    \item Sistem generiše obaveštenja menadžeru o artiklima čija je zaliha ispod minimuma i predlozima od veterinara za artikle koji ranije nisu naručivani.
    \item Menadžer otvara obaveštenja i pristupa listi artikala.
    \item Menadžer bira artikal koji želi da naruči. Ako menadžer želi da doda artikle koji nisu među obaveštenjima, ide se u alternativni tok A1.
    \item Sistem, po potrebi, predlaže količinu na osnovu istorije potrošnje.
    \item Sistem proverava budžet i prikazuje upozorenje ako je prekoračen. Ako narudžbina premašuje budžet ide se u alternativni tok A3.
    \item Menadžer potvrđuje ili menja količine. Ako artikal koji želi da naruči ne postoji u katalogu, ide se u alternativni tok A2.
    \item Koraci 3-6 se ponavljaju za sledece artikle, dok menadzer ne zavrsi izbor artikala.
    \item Menadžer kreira narudžbenicu, šalje je dobavljaču i sistem je evidentira.
\end{enumerate}
\indent \textbf{Alternativni tok:}
\begin{enumerate}
    \item[A1.]\textbf{Menadžer želi da doda artikle koji nisu u obaveštenjima} 
        \begin{enumerate}
            \item[1.] Menadžer otvara katalog artikala.
            \item[2.] Menadžer bira dodatne artikle za narudžbenicu.
            \item[3.] Sistem dodaje izabrane artikle u listu narudžbine.
            \item[4.] Tok se vraća na glavni tok, korak 4.
        \end{enumerate}
    \item[A2.]\textbf{Artikal koji želi da naruči ne postoji u katalogu}
        \begin{enumerate}
            \item[1.] Menadžer bira opciju ručnog dodavanja artikla.
            \item[2.] Menadžer unosi naziv, podatke o dobavljaču, cenu i druge parametre.
            \item[3.] Sistem dodaje novi artikal u privremenu listu narudžbine.
            \item[4.] Tok se vraća na glavni tok, korak 6 (menadžer sada menja ili potvrđuje količinu).
        \end{enumerate}
    \item[A3.]\textbf{Narudzbina premasuje budzet}
        \begin{enumerate}
            \item[1.] Sistem prikazuje upozorenje da bi narudžbina premašila dozvoljeni budžet.
            \item[2.] Menadžer menja količine ili uklanja artikle iz narudžbine.
            \item[3.] Sistem ažurira narudžbinu i ponovo proverava budžet: \\Ako je budžet sada u redu, tok se vraća na glavni tok, korak 6. \\Ako menadžer odustane, slucaj upotrebe se završava bez evidentiranja narudžbine.
        \end{enumerate}  
\end{enumerate}

\subsubsection{Prijem robe}
\indent \textbf{Kratak opis} - Menadžer prima naručenu robu u kliniku i ažurira stanje zaliha u katalogu, uključujući beleženje količina i datuma prijema. \\
\indent \textbf{Akteri} - Menadzer\\
\indent \textbf{Preduslovi} - Narudžbenica je kreirana i roba je stigla u kliniku. Menadžer je ulogovan u sistem.\\
\indent \textbf{Postuslov} - Katalog dostupnih lekova i materijala je ažuriran. Sistem evidentira primljene količine i datume prijema. \\
\indent \textbf{Glavni tok:}
\begin{enumerate}
    \item Menadzer otvara katalog na svom nalogu.
    \item Sistem prikazuje listu narudžbenica spremnih za prijem.
    \item Menadžer bira narudžbenicu koju prima.
    \item Menadžer unosi stvarno primljene količine za artikal.
    \item Sistem proverava da li unete količine odgovaraju naručenim i upozorava u slučaju odstupanja. Ako je primljena manja kolicina od naručene, ide se u alternativni tok A1. Ako je primljena veca kolicina od naručene, ide se u alternativni tok A2.
    \item Sistem ažurira stanje zaliha u katalogu i automatski evidentira datum prijema i primljene količine.
    \item Koraci 4-6 se ponavljaju sa sledeci artikal. Ukoliko su svi dostavljeni artikli evindetirani, menadzer zavrsava prijem robe.
\end{enumerate}
\indent \textbf{Alternativni tok:}
\begin{enumerate}
    \item[A1.]\textbf{Stiglo je manje robe nego naruceno} 
    \begin{enumerate}
        \item Sistem prikazuje upozorenje o manjku robe.
        \item Menadžer može naručiti razliku.
        \item Sistem ažurira narudžbenicu.
        \item Tok se vraća na glavni tok, korak 4 (za nastavak unosa sledećeg artikla).
    \end{enumerate}
    \item[A2.] \textbf{Stiglo je vise robe nego naruceno} 
    \begin{enumerate}
        \item Sistem prikazuje upozorenje o višku robe.
        \item Menadžer bira opciju da ili dodati višak u katalog kao dostupnu zalihu ili da kontaktirati dobavljača radi korekcije
        \item Sistem ažurira stanje zaliha po odluci menadžera.
        \item Tok se vraća na glavni tok, korak 4 (za nastavak unosa sledećeg artikla).
    \end{enumerate}
\end{enumerate}
