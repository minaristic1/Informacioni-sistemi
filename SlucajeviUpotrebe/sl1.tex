\subsection{Upravljanje terminima i zakazivanje pregleda} 
\indent Upravljanje terminima i zakazivanje pregleda su slučajevi upotrebe u kojima vlasnik ljubimca ima mogućnost da zakaže termin online, preko telefona i uživo u klinici. Administrativni radnik vodi evidenciju o slobodnim terminima i zakazuje termine preko telefonskog poziva i uživo u klinici, menja postojeće termine i otkazuje termine.


\begin{figure}[h]
    \centering
    \includegraphics[width=0.8\linewidth]{dijagramSlu.png}
    \caption{Slučaj upotrebe: Upravljanje terminima i zakazivanje pregleda}
    \label{fig:placeholder}
\end{figure}

\subsubsection{Slučaj upotrebe: Online zakazivanje}
\indent \textbf{Kratak opis} - Vlasnik ljubimca popunjava formular online sa potrebnim podacima i bira slobodan termin za odgovarajući pregled.\\
\indent \textbf{Akteri} - Vlasnik ljubimca \\
\indent \textbf{Preduslovi} - Vlasnik ljubimca ima pristup internetu, sistem je u funkiciji. \\
\indent \textbf{Postuslovi} - Vlasnik ljubimca je zakazao pregled, baza podataka je ažurirana. \\
\indent \textbf{Glavni tok:} 
\begin{enumerate}
 \item [1.] Vlasnik ljubimca otvara stranicu za zakazivanje pregleda.
 \item[2.] Sistem prikazuje formular za zakazivanje termina.
 \item[3.] Vlasnik ljubimca bira datum ili vremenski period za željeni pregled.
 \item[4.] Vlasnik ljubimca popunjava formular sa potrebnim podacima.
 \item[5.] Vlasnik ljubimca potvrđuje unos podataka na dugme potvrdi.
 \item[6.] Sistem validira unete podatke.
 \item[7.] Sistem čuva unete podatke.
 \item[8.] Sistem ažurira bazu podataka sa slobodnim terminima.
 \item[9.] Sistem obaveštava vlasnika ljubimca putem mejla da je uspešno zakazo termin.
\end{enumerate}
\indent \textbf{Alternativni tok:}
\begin{enumerate}
 \item [A1.]\textbf{Nema slobodnog termina za željeni period.} Ukoliko nema slobodnih termina za odabrani datum ili period u koraku 3 sistem izbacuje poruku da nema slobodnih termina za izabrani datum ili period. Proces se nastavlja u koraku 2 glavnog toka.
 \item [A2.]\textbf{Neuspešna validacija podataka.} Ukoliko u koraku 6 sistem naiđe na neispravno popunjeno polje formulara, sistem obeležava polje koje je potrebno ispraviti i obaveštava vlasnika ljubimca. Vlasnik ljubimca ispravlja unos. Proces se nastavlja u koraku 5 glavnog toka.
 \item[A3.]\textbf{Vlasniku ljubimca nije stigao mejl sa obaveštenjem o zakazivanju.} Ukoliko sistem nije poslao mejl o potvrdi zakazanog termina znači da je neuspešno zakazan termin. Proces se završava.
\end{enumerate}
\indent \textbf{Dodatne informacije} - Potrebni podaci za zakazivanje pregleda su ime i prezime vlasinka ljubimca, broj telefona vlasnika ljubimca, mejl vlasnika ljubimca, vrsta ljubimca, željeni pregled (vakcinacija, intervencija, operacija, kontrolni pregled,...). \\

\subsubsection{Slučaj upotrebe: Zakazivanje termina}
\indent \textbf{Kratak opis} - Administrativni radnik zakazuje termine pregleda u dogovoru sa vlasnikom ljubimca i ažurira tabele i bazu podataka. \\
\indent \textbf{Akteri} - Administrativni radnik, vlasnik ljubimca \\
\indent \textbf{Preduslovi} - Administrativni radnik je identifikovan i autorizovan za korišćenje sistema. Sistem je u funkciji. \\
\indent \textbf{Postuslovi} - Pregled je zakazan. Baza podataka je ažurirana. \\
\indent \textbf{Glavni tok:} 
\begin{enumerate}
 \item [1.] Vlasnik ljubimca se obraća administrativnom radniku radi zakazivanja pregleda.
 \item[2.] Administrativni radnik otvara tabelu za zakazivanje termina.
 \item[3.] Administrativni radnik unosi u tabelu potrebne podatke.
 \item[4.] Administrativni radnik otvara tabelu sa slobodnim terminima.
 \item[5.] Administrativni radnik pregleda dostupne termine.
 \item[6.] Administrativni radnik u dogovoru sa vlasnikom ljubimca bira slobodan termin.
 \item[7.] Administrativni radnik unosi datum i vreme u tabelu za zakazivanje termina.
 \item[8.] Administrativni radnik potvrđuje zakazani termin. 
 \item[9.] Administrativni radnik ažurira tabelu sa slobodnim terminima.
 \item[10.] Administrativni radnik ažurira bazu podataka.
\end{enumerate}
\indent \textbf{Alternativni tok:}
\begin{enumerate}
 \item [A1.]\textbf{Nema slobodnih termina u željenom periodu} Administrativni radnik informiše vlasnika ljubimca da nema slobodnih termina. Predlaže vlasniku ljubimca najbliže slobodne datume. Vlasnik ljubimca bira novi termin i proces se vraća na korak 7 glavnog toka.
 \item[A2.]\textbf{Tehnički problem pri upisu termina.} Administrativni radnik ne može da završi zakazivanje i pokušava ponovo malo kasnije.
 \item[A3.]\textbf{Nijedan slobodan termin ne odgovara vlasniku.} Ukoliko u koraku 6 ne dođe do dogovora izmedju administrativnog radnika i vlasnika ljubimca, administrativni radnik briše prethodno unete podatke. Proces se završava.
\end{enumerate}
\indent \textbf{Dodatne informacije} - Potrebni podaci za zakazivanje pregleda su ime i prezime vlasinka ljubimca, broj telefona vlasnika ljubimca, mejl vlasnika ljubimca, vrsta ljubimca, željeni pregled (vakcinacija, intervencija, operacija, kontrolni pregled,...). Vlasnik ljubimca može da zakaže pregled uživo u klinici ili putem telefona. \\
